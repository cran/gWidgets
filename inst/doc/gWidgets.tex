\documentclass[12pt]{article}
\newcommand{\VERSION}{0.0-7}
%\VignetteIndexEntry{gWidgets}
%\VignettePackage{gWidgets}
%\VignetteDepends{methods}

\usepackage{times}              % for fonts
\usepackage[]{geometry}
\usepackage{mathptm}            % for math fonts type 1
\usepackage{graphicx}           % for graphics files
\usepackage{floatflt}           % for ``floating boxes''
%%\usepackage{index}
\usepackage{relsize}            % for relative size fonts
\usepackage{amsmath}            % for amslatex stuff
\usepackage{amsfonts}           % for amsfonts
\usepackage{url}                % for \url,
\usepackage{hyperref}
\usepackage{color}
\usepackage{fancyvrb}
\usepackage{fancyhdr}
\usepackage{jvfloatstyle}       % redefine float.sty for my style. Hack


%% squeeze in stuff
\floatstyle{jvstyle}
\restylefloat{table}
\restylefloat{figure}
\renewcommand\floatpagefraction{.9}
\renewcommand\topfraction{.9}
\renewcommand\bottomfraction{.9}
\renewcommand\textfraction{.1}
\setcounter{totalnumber}{50}
\setcounter{topnumber}{50}
\setcounter{bottomnumber}{50}

%% Fill these in
\pagestyle{fancy}
\usepackage{fancyhdr}
\pagestyle{fancy}
\fancyhf{}
\fancyhead[L]{\RCode{gWidgets}}
\fancyhead[C]{}
\fancyhead[R]{\sectionmark}
\fancyfoot[L]{}
\fancyfoot[C]{- page \thepage\/ -}
\fancyfoot[R]{}
\renewcommand{\headrulewidth}{0.1pt}
\renewcommand{\footrulewidth}{0.0pt}

%% My abbreviations
\newcommand{\RCode}[1]{\texttt{#1}}
\newcommand{\RFunc}[1]{\texttt{#1()}}
\newcommand{\RPackage}[1]{\textbf{#1}}
\newcommand{\RArg}[1]{\texttt{#1=}}
\newcommand{\RListel}[1]{\texttt{\$#1}}


\newenvironment{RArgs}{\begin{list}{}{}}{\end{list}}


\usepackage{/home/verzani/R/lib/R/share/texmf/Sweave}
\begin{document}
\thispagestyle{plain}
\title{Examples for gWidgets}

\author{John Verzani, \url{gWidgetsRGtk@gmail.com}}
\maketitle

\section*{Abstract:}
Examples for using the \RPackage{gWidgets} package are presented.  The
\RPackage{gWidgets} API is intended to be a cross platform means
within an R session to interact with a graphics toolkit.  Currently,
there are two available toolkits. The GTK toolkit is implemented via
the \RPackage{gWidgetsRGtk2} package which in turn uses the
\RPackage{RGtk2} package.  JAVA can be used via the
\RPackage{gWidgetsrJava} package, which in turn calls
\RPackage{rJava}. The \RPackage{gWidgetsRGtk2} implementation is much
more complete, as the \RPackage{gWidgetsrJava} package is lacking many
features and must run from within a JGR (\url{http://www.rosuda.org})
session.


The API is inteneded to faciliatate the task of writing basic GUIs, as
it simplifies many of the steps involved in setting up widgets and
packing them into containers. Although not nearly as powerful as any
individual toolkit, the \RPackage{gWidgets} API is suitable for many
tasks or as a rapid prototyping tool for more complicated
applications. The examples contained herein illustrate that quite a
few things can be done fairly easily with more complicated
applications being pieced together in a straightforward manner.  To
see a fairly complicated application built using \RPackage{gWidgets},
install the \RPackage{pmg} GUI
(\url{http://www.math.csi.cuny.edu/pmg}), which is on CRAN.


\setcounter{tocdepth}{3}
\tableofcontents

\section{Background}
The \RCode{gWidgets} API is intended to be a cross-toolkit API for
working with GUI objects. It is based on the iwidgets API of Simon
Urbanek, with improvement by Philippe Grosjean, Michael Lawrence,
Simon Urbanek and John Verzani.  This document focuses on the more
complete toolkit implementation provided by the
\RPackage{gWidgetsRGtk2} package. [Occasional differences with
\RPackage{gWidgetsrJava} are pointed out in braces.] The GTK toolkit
is interfaced via the \RPackage{RGtk2} package of Michael Lawrence,
which in turn is derived from Duncan Temple Lang's \RPackage{RGtk}
package. The excellent \RPackage{RGtk2} package opens up the full
power of the GTK2 toolkit, only a fraction of which is available
though \RPackage{gWidgetsRGtk2}.

The \RCode{gWidgets} API is still in the formative stages and likely will
change as more people use it and offer suggestions for improvement.

This document supplements the man pages by providing more detailed
examples. The man pages contain more specific information. See the 
page \RCode{gWidgets-package} for a listing of the available man pages.

This document is a vignette. As such, the code displayed is available
within an R session through the command
\RCode{edit(vignette("gWidgets"))}. 

\section{Installation}
In case you are reading this vignette without having installed
gWidgets, here are some instructions. This focuses on installing
\RPackage{gWidgetsRGtk2}. 

Installing \RCode{gWidgets} with the \RCode{gWidgetsRGtk2} pakcage
requires two steps: installing the GTK libraries and installing the R
packages.



\subsection{Installing the GTK libraries}

The \RCode{gWidgetsRGtk2} provides a link between \RCode{gWidgets} and
the GTK libraries through the \RCode{RGTk2} package.  \texttt{RGtk2}
requires relatively modern versions of the GTK libraries (2.8.0 or
higher). These may need to be installed or upgraded on your system.

In case of \textbf{Windows} do this:

\begin{enumerate}
\item 
 Download the files from
\href{http://gladewin32.sourceforge.net/modules/wfdownloads/visit.php?lid=102}{http://gladewin32.sourceforge.net/modules/wfdownloads/visit.php?lid=102}

\item run the resulting file.  This is an automated installer which will walk
you through the installation of the Gtk2 libraries.
\end{enumerate}

In Linux, you may or may not need to upgrade the GTK libraries
depending on your distribution.

For Mac OS X, this author has installed GTK libraries from source on an older
10.3.9 machine using Apple's X11 server. It may be possible in 10.4 to
install using fink or Darwin ports, or it may be possible to run the
native GTK libraries. None of these has been tested.

There are more details on RGtk2 at 
\href{http://www.ggobi.org/rgtk2}{RGtk2's home page}.


\subsection{Install the R packages}

The following R packages are needed: \texttt{RGtk2},
\texttt{cairoDevice}, \texttt{gWidgets}, and
\texttt{gWidgetsRGtk2}. Install them in this order, as some depend on
others to be installed first. All can be downloaded from CRAN.

These can all be installed by following the dependencies for
\RCode{gWidgetsRGtk2}. The following command will install them all if
you have the proper write permissions:

\begin{verbatim}
   install.packages("gWidgetsRGtk2", dep = TRUE)
\end{verbatim}

It may be necessary  to adjust the location where the libraries will be
installed if you do not have the proper permissions.

For MAC OS X, the packages are provided as source, not the default
``mac.binary.'' Use the \RCode{type=} argument, as follows:
\begin{verbatim}
> install.packages("gWidgetsRGtk2", dep = TRUE,type = "source" )
\end{verbatim}

On occasion, newer versions are available from
\href{http://www.math.csi.cuny.edu/pmg}{gWidgets's website}. To
install from here add the \texttt{repos=} argument, as follows:

\begin{verbatim}
> install.packages("gWidgetsRGtk2",dep = TRUE, repos = "http://www.math.csi.cuny.edu/pmg")
\end{verbatim}


[The \RPackage{gWidgetsrJava} package is installed similarly. However,
it requires \RPackage{rJava} and \RPackage{JGR} and friends to work
properly. These should install through the dependencies, so
\begin{verbatim}
> install.packages("gWidgetsrJava", dep = TRUE)
\end{verbatim}
should do it.]


\section{Loading gWidgets}
We  load the \RCode{gWidgets} package, using the \RCode{gWidgetsRGtk2}
toolkit, below:
following 
\begin{Schunk}
\begin{Sinput}
> options(guiToolkit = "RGtk2")
> require(gWidgets)